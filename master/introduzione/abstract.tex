\chapter*{Abstract}
\markboth{ABSTRACT}{ABSTRACT}
\addcontentsline{toc}{chapter}{\numberline{}Abstract}
\setcounter{footnote}{0}

In the last three decades CPU performance improvements have been achieved exclusively by increasing the clock frequency. This was been made possible by the increasing number of integrated circuits present in the silicon die (process largely known as Moore's law~\cite{moore}). Moore's law allowed, from one side, the integration of additional logic inside the chip; while the first CPUs were rather simple, later they have slowly evolved to include Floating Point (FPU) and Single Instruction Multiple Data (SIMD) units, complex branch prediction modules, etc. Besides, an increasing density has lead to a reduction of the paths inside the chip, and thus the possibility to sustain higher clock frequencies.

At the moment, this exponential increment in performance is reaching a stop. Firstly, for single threaded CPUs the increasing number of transistor has a limited effect on performance. While in the past CPUs could be improved by adding more logic units, today it becomes more difficult to exploit all the additional hardware components at their full potential. Secondly, memory speeds do not increase at the same pace as CPUs and thus cache misses become more and more expensive. As a result the area dedicated to branch prediction, out of order execution logic and larger caches has grown. Additionally, higher densities translate into higher power consumption.

For all these reasons, CPUs designers are exploring alternative performance improvement solutions rather than just increasing the clock frequency. In particular, one of the solutions focuses on the study of multi-core architectures that include several SIMD units, thus exposing the intrinsic instruction level parallelism. A new, more radical, project is the Cell processor, or Cell Broadband Engine Architecture (CBEA). Rather than slowly evolving towards multi-core architectures, the Cell processor has been designed from scratch keeping in mind all these concepts.

The CBEA is the result of four years of development from the STI consortium, started by Sony, Toshiba and IBM to realize a high performance MultiProcessor System on Chip (MPSoC). Sony uses the Cell as compute unit of the PlayStation3, Toshiba plans to include it in its HD TV products, while IBM uses it in its high performance servers for scientific computing.

The Cell is an heterogeneous MPSoC made of 12 compute units (cores), connected through an Element Interconnect Bus (EIB): a Power Processing Element (PPE), 8 Synergistic Processing elements (SPEs), a Memory Interface Controller (MIC) and 1 bus controllers with two external interfaces (IOIF0 and IOIF1). The PPE is a 64-bit CPU based on the IBM Power architecture~\cite{PowerArchitecture}. It has 32-KB of level-1 and 512-KB of level-2 cache space clocked at 3.2 GHz. The SPEs are 128-bit SIMD units, each of which has a 256-KB Local Store (LS) clocked at 3.2 GHz. The Memory Flow Controller (MFC) inside the SPEs takes care of managing LS memory and the DMA module, this last one represents the only support to move data between LS and system memory. The MIC provides access to the system memory through the RAMBUS XDR RAM. Each SPE can support up to 51.2 GB/s. For this reason, a crucial role in delivering high performance is played by the EIB component which must support the system's full potentials.

The evolution of SoC(s) towards MPSoC(s) moves further the focus on the interconnect element inside the chip, which thus becomes the main performance bottleneck. According to the International Roadmap for Semiconductors, the increasing clock frequency in the next 10 years will make possible to manufacture chips integrating more than a billion of transistors and working at clock frequencies of 10 GHz~\cite{ITRS}. It is not possible to take full advantage of such technological improvements using shared bus systems, as it happens nowadays. In fact, long wires running from one side to another of the chip to connect the different logic units impose an upper limit on the maximum clock frequency achievable, due to the longer critical path. For all these reasons, Network on Chip (NoC) architecture are becoming more and more popular in the community. Additionally, the adoption of this technology brings the following benefits:

\begin{itemize}
	\item[$\circ$] It is possible to reuse many of the technologies that have been developed for packed-switched networks, adapting them to on chip connections.
	\item[$\circ$] It is possible to reach a higher degree of flexibility through
	\begin{itemize}
		\item[-] Modularity (extensive use of parametric functional blocks)
		\item[-] Reconfigurability (functional blocks can be connected to match topologies of different use cases).
	\end{itemize}
	\item[$\circ$] It is possible to easily integrate Intellectual Properties\footnote{IP here means an HW or SW component usable in different heterogeneous systems.} (IP cores) developed from third parties, provided that they comply to the same communication interface with the external components.
\end{itemize}

The objective of this thesis is the evaluation of the CBEA EIB performance and the study of the benefits achievable by replacing it with an alternative packet switched solution. The work is structured in three parts. First, we extend the CellSim~\cite{Cellsim} simulation framework, developed by BSC~\cite{BSC} using the UniSim library~\cite{Unisim}, to benchmark the EIB component while varying its architectural parameters. Second, we integrate Cellsim with XPIPES~\cite{XPIPES}, a NoC cycle accurate component library written in SystemC. XPIPES has been jointly developed by University of Cagliari, University of Bologna~\cite{Bologna} and University of Stanford~\cite{Stanford}. Third, we benchmark the interconnect technologies under study, using a selected number of software benchmarks, and compare their performance when varying the topology as well as the architectural parameters.

\subsection*{Structure of the thesis}
The first chapter of the thesis presents the Cell Broadband Engine Architecture starting from its design up until the production of the first generation of Cell processors. More specifically, we focus on the new feature introduced to overcome the performance limitations that affect many architecture currently in use such as, for example, symmetric multiprocessors. 

The second chapter presents the Cellsim simulator, here used for the performance profiling of the interconnect infrastructures studied in the thesis, EIB and $\times$pipes. We also describe the UNISIM framework, used to build the Cellsim simulator.

The third chapter presents the $\times$pipes NoC architecture, with focus on the communication protocol used (OCP) to exchange messages among the IP cores.

The fourth chapter presents the element interconnect bus simulation model, developed in collaboration with BSC, and the additional infrastructure developed for its performance profiling. 

The fifth chapter presents the problems related to the integration of the SystemC $\times$pipes simulation model in the Cellsim simulator. Focusing on the network interface model used to make Cellsim IP cores compatible with $\times$pipes.

Finally, the sixth chapter presents the results obtained through apposite software benchmarks. Performance are measured using the number of execution cycles and the average message latency in the interconnect as metrics. 
