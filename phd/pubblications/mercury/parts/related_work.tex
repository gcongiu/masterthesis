%!TEX root = ../main.tex
\section{Related Work}
\label{sec: related_work}

In the past researchers have tried to alleviate the I/O performance gap problem by analyzing I/O patterns and exploiting their knowledge to guide I/O using, for example, data prefetching. Tran and Reed~\cite{TranR04} presented an automatic time series modelling and prediction framework for adaptive I/O prefetching, named TsModeler. They combined ARIMA and Markov models to describe temporal and spatial behaviour of I/O patterns at file block level. TsModeler was integrated with the experimental file system PPFS2 and used to predict future accesses. %tested against a selected physics code. Several characteristics, such as execution time improvements and cache miss reduction over different hardware configurations, are considered in the experiments. The results show that execution time can be reduced by the 30\% in some cases and cache misses can be reduced up to three order of magnitude. 
He et al.~\cite{HEBTAGGMCS13} proposed a pattern detection algorithm, based on the sliding window algorithm in LZ77 as base for building Markov models of I/O patterns at file block level. The model was afterwards used by a FUSE based file system to carry out prefetching. Chang and Gibson~\cite{ChangG99}, unlike previous works, did not build mathematical models but instead used speculative execution of the application code to guide data prefetching.  

Other works tried to bring the same idea to higher level I/O libraries such as MPI I/O, HDF5 or PnetCDF to take advantage of the richer semantic, data dependencies and layout information. Chen, Byna, Sun, Thakur and Gropp~\cite{ChenBSTG08} proposed a pre-execution based prefetching approach to mask I/O latency. They provided every MPI process with a thread that runs in parallel and takes responsibility for prefetching future required data. Prefetching in the parallel thread was enabled via speculative execution of the main process code. %Results, with PBench running on top of NFS and PVFS as file systems backend, show execution time reduction and sustained bandwidth improvements. 
The same authors in~\cite{BynaCST08} proposed to exploit parallel prefetching using a client-side, thread based, collective prefetching cache layer for MPI I/O. The cache layer used I/O pattern information, in the form of I/O signatures, together with run time I/O information to predict future accesses. %Experimental results show sustained bandwidth improvements even in this case. 
Chen and Roth~\cite{ChenR10} took inspiration from the collective I/O optimization enabled by ROMIO to design a collective I/O data prefetching mechanism that exploited global I/O knowledge. %They compare the sustained bandwidth speed-up of individual prefetching with collective prefetching for a parallel benchmarking tool using PVFS2, and demonstrate that the latter performs better than the former by over two fold on average. 
He, Sun and Thakur~\cite{HEST12} proposed to analyze high level data dependencies exposed in PnetCDF, accumulate this knowledge building data dependency graphs and finally use them to perform prefetching. 

VanDeBogart, Frost and Kohler have previously used the Linux advice API to build a prefetching library~\cite{VanDeBogartFK09} for programmers to use. Prost et al. integrated the GPFS hint functionalities in the ROMIO ADIO driver for GPFS~\cite{ProstTHJK01}. In this context they exploit data type semantic in file views to prefetch parts of the file that will be soon accessed. 

In contrast to previous works, we do the following things differently. We do not try to automatically build mathematical models of I/O patterns and use them to accurately generate prefetching requests nor do we speculatively execute the application binary. In fact, we believe that users and administrators have the best understanding about the applications and their systems, and can exploit their knowledge and expertise to improve the storage system performance. We demonstrate that experienced users with a deep knowledge of their applications I/O behavior can convert non-optimal I/O patterns, in particular small random reads, into patterns that can be adapted to the underlying file system characteristics, and therefore give optimal performance. %In this work we focus on providing the infrastructure that enables users to access file system specific interfaces for guided I/O without modifying applications and hiding the intrinsic complexity that such interfaces introduce.  %and Indeed, we found that some I/O patterns look sequential if we observe them at a granularity coarser than the single request. 
Furthermore, previously described approaches are not suitable for small random read patterns since they rely on accurate knowledge of I/O behaviour % and the assumption that this remains unchanged over multiple runs of the application. Moreover, even if I/O patterns remain unchanged, the prefetching of random requests will still degrade the performance of the storage subsystem. 
to prefetch every single request one after the other. This still degrades the storage system performance due to the large number of I/O requests and seek operations hitting the storage devices. On the other hand, by using the POSIX advice and GPFS hints APIs, we can prefetch the region of the file that will be accessed and filter random requests using the cache. %Data prefetching allows us to increase the I/O bandwidth, reduce the number of I/O requests reaching the back-end storage devices and the execution time of applications. 

In this work we focus on providing the infrastructure that enables users to access file system specific interfaces for guided I/O without modifying applications and hiding the intrinsic complexity that such interfaces introduce. %.In this work we focus our attention on applications that use the POSIX I/O interface and have no access to the optimizations enabled by MPI I/O. Our idea is to exploit post morten I/O pattern analysis to generate advice and hints transparently for every file. 
