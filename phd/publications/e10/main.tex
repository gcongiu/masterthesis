\documentclass[10pt, conference, compsocconf]{IEEEtran}
\ifCLASSINFOpdf
  \usepackage[pdftex]{graphicx}
  % declare the path(s) where your graphic files are
  % \graphicspath{{../pdf/}{../jpeg/}}
  % and their extensions so you won't have to specify these with
  % every instance of \includegraphics
  % \DeclareGraphicsExtensions{.pdf,.jpeg,.png}
\else
  % or other class option (dvipsone, dvipdf, if not using dvips). graphicx
  % will default to the driver specified in the system graphics.cfg if no
  % driver is specified.
  \usepackage[dvips]{graphicx}
  % declare the path(s) where your graphic files are
  % \graphicspath{{../eps/}}
  % and their extensions so you won't have to specify these with
  % every instance of \includegraphics
  % \DeclareGraphicsExtensions{.eps}
\fi

\usepackage{listings}
\usepackage{color}
\usepackage{xcolor}
\usepackage{caption}
\usepackage{subcaption}
\usepackage[cmex10]{amsmath}
\usepackage{courier}
\usepackage[T1]{fontenc}
\usepackage{url}
\usepackage{array, booktabs}
\usepackage{todonotes}
\hyphenation{guided-io}
\newcommand{\codeword}{\texttt}
\newcommand{\ra}[1]{\renewcommand{\arraystretch}{#1}}
\renewcommand\IEEEkeywordsname{keywords}
\newcolumntype{M}[1]{>{\centering\arraybackslash}m{#1}}
\begin{document}

\title{Improving Collective I/O Performance Using Non-Volatile Memory Devices}
\author{
%\IEEEauthorblockN{Giuseppe Congiu}
%\IEEEauthorblockA{Emerging Technology Group\\
%Seagate Systems UK Ltd\\
%Havant, United Kingdom\\
%Email: giuseppe.congiu@seagate.com}
%\and
%\IEEEauthorblockN{Sai Narasimhamurthy}
%\IEEEauthorblockA{Emerging Technology Group\\
%Seagate Systems UK Ltd\\
%Havant, United Kingdom\\
%Email: sai.narasimhamurthy@seagate.com}
%\and
%\IEEEauthorblockN{Andr\'{e} Brinkmann}
%\IEEEauthorblockA{Zentrum f\"ur Datenverarbeitung\\
%Johannes Gutenberg-Universit\"{a}t\\
%Mainz, Germany\\
%Email: brinkman@uni-mainz.de}}

\IEEEauthorblockN{Giuseppe Congiu\IEEEauthorrefmark{1}, Sai Narasimhamurthy\IEEEauthorrefmark{1}, Tim S\"u\ss{}\IEEEauthorrefmark{2}, Andr\'e Brinkmann\IEEEauthorrefmark{2}}
\IEEEauthorblockA{\IEEEauthorrefmark{1}Emerging Technology Group Seagate Systems Ltd Havant, United Kingdom\\
Email: \{giuseppe.congiu, sai.narasimhamurthy\}@seagate.com}
\IEEEauthorblockA{\IEEEauthorrefmark{2}Zentrum f\"ur Datenverarbeitung Johannes Gutenberg-Universit\"{a}t Mainz, Germany\\
Email: \{t.suess, brinkman\}@uni-mainz.de}}

\maketitle

% What is the abstract word limit for this conference?
\begin{abstract}
Collective I/O is a parallel I/O technique designed to deliver high performance data access to scientific applications running on high-end computing clusters. In collective I/O, write performance is highly dependent upon the storage system response time and limited by the slowest writer. The storage system response time in conjunction with the need for global synchronisation, required during every round of data exchange and write, severely impacts collective I/O performance. Future Exascale systems will have an increasing number of processor cores, while the number of storage servers will remain relatively small. Therefore, the storage system concurrency level will further increase, worsening the global synchronisation problem. Nowadays high performance computing nodes also have access to locally attached solid state drives, effectively providing an additional tier in the storage hierarchy. Unfortunately, this tier is not always fully integrated. In this paper we propose a set of MPI-IO hints extensions that enable users to take advantage of fast, locally attached storage devices to boost collective I/O performance by increasing parallelism and reducing global synchronisation impact in the ROMIO implementation. We demonstrate that by using local storage resources, collective write performance can be greatly improved compared to the case in which only the global parallel file system is used, but can also decrease if the ratio between aggregators and compute nodes is too small.
\end{abstract}
\begin{IEEEkeywords}
MPI-IO; Collective I/O; Non-Volatile Memory; HPC;
\end{IEEEkeywords}
\vspace{-1mm}
%!TEX root = ../main.tex
\section{Introduction}
\label{sec: introduction}

The gap between hard disk drives' (HDDs) performance and processors' computing power, better known as the I/O performance gap problem, represents a serious scalability limitation especially for scientific applications running on High End Computing (HEC) clusters. 
Parallel File Systems (PFSs) such as Lustre~\cite{Braam02}, PVFS~\cite{CarnsLRT} and GPFS~\cite{SchmuckH02}, just to mention a few, try to bridge this gap by striping files across multiple storage devices and providing multiple parallel data paths to increase the aggregate I/O bandwidth and the number of IOPS. The ROMIO middleware\footnote{Implementation of MPI I/O specifications from Argonne National Laboratory included in MPICH package (http://www.mpich.org/).} implements extensions to the POSIX I/O interface typically provided by PFSs that result in a richer parallel I/O interface, and through the Abstract Device I/O (ADIO) driver~\cite{ThakurGL96} enables transparent file access optimizations based on two-phase I/O and data sieving to adapt I/O patterns to the characteristics of the underlying file system~\cite{ThakurGL99}~\cite{Ying08}~\cite{ProstTHKW00}.
%
Nevertheless, as Carns et al. have pointed out in their study~\cite{CarnsHABLLR11} most of the scientific applications running on big clusters today still use the POSIX I/O interface to access their data. %Furthermore, it has also been ascertained that using POSIX I/O to access non-contiguous regions of files causes extremely poor performance in the case of PFSs~\cite{ChingCLP06}. Indeed, PFSs provide best I/O bandwidth performance for large contiguous requests while they typically provide only a fraction of the maximum bandwidth in the opposite case. This is primarily due to the high number of remote procedure calls generated by the file system clients that overwhelms I/O servers and the resulting high number of HDDs' head movements in every I/O target (seek overhead). 

Currently there is no available solution to overcome limitations caused by non-optimal file I/O patterns generated by applications, except to re-write them. In this context, the Linux kernel provides users with the capability to communicate access pattern information to the local file system through the \texttt{posix\_fadvise()}~\cite{AdviseAPI} system call. The file system can use this information to improve page cache efficiency, for example, by prefetching (or releasing) data that will (or will not) be required soon in the future or by disabling read-ahead in the case of random read patterns. However, \texttt{posix\_fadvise()} is barely used in practice and has intrinsic limitations that discourage its employment in real applications. 
%TODO: is there a citation for this 'barely' claim?

The two most used PFSs in HEC clusters nowadays, IBM GPFS and Lustre, are both POSIX compliant. Nevertheless, neither of them support the POSIX advice mechanism previously described. GPFS compensates for the lack of POSIX advice support through a hints API that users can access by linking their programs against a service library. Hints are passed to GPFS through the \texttt{gpfs\_fcntl()}~\cite{GPFSHINTS} function and can be used to guide prefetching (or releasing) of file blocks in the page pool\footnote{GPFS pinned memory used for file system caching.}. However, unlike POSIX advice, GPFS hints %are not mandatory and 
can be discarded by the file system if certain requirements are not met. Lustre, on the other hand, does not provide any client side mechanism similar to GPFS hints or POSIX advice. Recently a new Lustre advice mechanism has been proposed by DDN during the Lustre User Group 2014 (LUG14) in Miami~\cite{Comer14}. The difference in the DDN approach is that it provides control over the storage servers (OSSs) cache instead of the file system client cache.
%TODO: why is your approach better? Yours has the penalty of moving the data over the wire into the client cache. Perhaps the approaches are complimentary, you address "I will want this data", they address "someone will want this data, but _they_ may not know it yet"

In this paper we propose and evaluate a novel guided I/O framework called "MERCURY"~\cite{mercury} able to optimize file access patterns at run-time through data prefetching using available hints mechanisms. MERCURY communicates file I/O pattern information to the file system on behalf of running applications using a dedicated process that we call \textit{Advice Manager}. In every node of the cluster, processes can access their files using an \textit{Assisted I/O library} that transparently forwards intercepted requests to the local \textit{Advice Manager}. This uses \texttt{posix\_fadvise()} and \texttt{gpfs\_fcntl()} to prefetch (or release) data into (or from) the client's file system data cache. The \textit{Assisted I/O library} controls for which files advice or hints should be given, while the \textit{Advice Manager} controls how much data to prefetch (or release) from each file. Monitored file paths and prefetching information are contained in a configuration file that can be generated either manually or automatically once the I/O behaviour of the target application is known. The configuration file mechanism allows us to decouple the specific hints API provided by the back-end file system from the generic interface exposed to the final user thus making our solution portable.

With this approach we are able to generate POSIX advice and GPFS hints for applications that do not use them but can receive a benefit from their use. We accomplish this asynchronously, %with very low overhead, 
and without any modification of the original application. We demonstrate that our approach is effective in improving the I/O bandwidth, reducing the number of I/O requests and reducing the execution time of a `ROOT' \footnote{Data analysis framework developed at CERN.}~\cite{root} based analytic application.

Additionally, we propose and evaluate a modification to the Linux kernel that makes it possible for Lustre, and in principle other networked file systems, to participate in activity triggered by the \texttt{posix\_fadvise()} system call, thus allowing it to take advantage of our guided I/O framework benefits.

The remainder of this paper is organised as follows. Section~\ref{sec: background} covers the background on file systems support to guided I/O interfaces (POSIX advice and GPFS hints), Section~\ref{sec: concept} presents concept, design and implementation of the MERCURY prototype highlighting the main contributions of the work. This section also describes the kernel modifications that enable POSIX advice on Lustre, Section~\ref{sec: evaluation} presents the evaluation of our prototype on three file systems: a local Linux ext4 file system, a GPFS file system and a Lustre file system, Section~\ref{sec: related_work} presents related works on data prefetching, and finally Section~\ref{sec: conclusion} presents conclusion and future work.    

%\todo[inline]{need to make clear that the configuration file enables user to specify prefetching information in the most generic way. Afterwards, this information is translated to match the corresponding backend file system hint interface}

%\todo[inline]{also need to add lustre to the mix of file systems. In this case there will be a dedicated section to the VFS patching to make POSIX\_FADV\_WILLNEED work with lustre}

%!TEX root = ../main.tex
\section{Collective I/O in ROMIO}
\label{sec: mpi-io}

ROMIO is a popular implementation of the MPI-IO specification developed at the Argonne National Laboratory and currently supported by MPICH as well as OpenMPI and other packages. ROMIO provides parallel I/O functionalities for different file systems through the Abstract Device I/O interface~\cite{ThakurGL96} (ADIO). Latest versions of ROMIO include support for Lustre~\cite{Ying08}, GPFS~\cite{ProstTHKW00}, PVFS~\cite{CarnsLRT} and others through a dedicated ADIO driver.

\subsection{Two Phase I/O}
\label{sec: ext2ph}

In ROMIO, the core component of collective I/O is the `two phase I/O', also known as `extended two phase algorithm' (ext2ph)~\cite{ThakurC96}. The ROMIO implementation for collective I/O consists of several steps as follows:
\begin{figure*}[!htb]
  \centering
  \includegraphics[width=0.95\textwidth]{figures/ext2ph}
  \caption{Collective I/O flow diagram for the write path in aggregators (non-aggregators neither receive nor write any data, just send it to aggregators). \codeword{MPI\_File\_write\_all()} invokes \codeword{ADIOI\_GEN\_WriteStridedColl()}. \codeword{ADIO\_WriteContig} is a macro that is replaced by \codeword{ADIOI\_GEN\_WriteContig()}. Performance critical functions for the collective I/O branch are highlighted in grey.}
  \label{figure: coll_io_impl}
\end{figure*}

\begin{enumerate}
\item All processes taking part in the I/O operation exchange access pattern information with each other. The access pattern information is represented by start and end offsets for the accessed region (disregarding holes that may be present). Once file offsets are available, every process works out how big the global accessed region in the file is by taking maximum and minimum among all. The resulting byte range is divided by the number of available aggregators to build the so called `file domains' (contiguous byte ranges accessed independently by every aggregator).
\item Every process works out which file domains (and thus aggregators) its local data belongs to. In doing so, every process knows which aggregators it has to send (receive) data to (from), if any.
\item Every aggregator works out which other processes' requests map to its file domain. Doing so every aggregator knows what processes need to receive (in case of reads) or send (in case of writes) data for that particular file domain.
\item Actual two phase I/O starts. In the case of writes, that we exclusively consider here (the read case is similar), every process sends its data to the right aggregators (data shuffle phase) while these write the data to the parallel file system (data I/O phase). Data is written in blocks of predefined size (collective buffer size). If the size of the collective buffer is smaller than the file domain, the file domain is broken down into multiple sub-domains which are written in different rounds of the ext2ph algorithm. In order to handle multiple rounds of data shuffle and I/O, additional access information is required. This is disseminated by every process (collectively) to aggregators at the beginning of the data shuffle phase.
\item Once all the data has been written, all the processes must synchronise and exchange error codes. This is necessary to guarantee that it is safe to free the memory buffers containing the data.
\end{enumerate}
Figure~\ref{figure: coll_io_impl} shows how the previous steps map to the collective I/O implementation for the write operation. The collective write function (\codeword{MPI\_File\_write\_all()}) in ADIO is implemented through \codeword{ADIOI\_GEN\_WriteStridedColl()}. This is responsible for selecting the most suitable I/O method between those available. For example, independent I/O is selected if the access requests are not interleaved. Nevertheless, users can always enforce collective I/O by setting the appropriate MPI-IO hint. The \codeword{ADIOI\_Exch\_and\_write()} function contains the ext2ph algorithm implementation, including data shuffle and write methods. At the beginning of the data shuffle (\codeword{ADIOI\_W\_Exchange\_data()}) we have the dissemination function (\codeword{MPI\_Alltoall()}) used to exchange information concerning which part of the data has to be sent during a particular round of two phase I/O. 

There are three main contributors to collective I/O performance: (\textbf{a}) global synchronisation cost; (\textbf{b}) communication cost; and (\textbf{c}) write cost. \codeword{MPI\_Allreduce()} and \codeword{MPI\_Alltoall()} account for the global synchronisation cost. When a process reaches them it has to wait for all the other processes to arrive before continuing. \codeword{MPI\_Waitall()} accounts for communication cost since every process first issues all the non-blocking receives (if any) and sends, and afterwards waits for them to complete (refer to the right part of the diagram in Figure~\ref{figure: coll_io_impl}). Finally, \codeword{ADIO\_WriteContig()} accounts for write cost.

\subsection{Collective I/O Hints}
\label{subsec: hints}

Collective I/O behaviour can be controlled by users through a set of MPI-IO hints. Users can control whether collective I/O should be enabled or disabled with \codeword{romio\_cb\_write} and \codeword{romio\_cb\_read}, for write and read operations respectively, how many aggregators should be used during a collective I/O operation with \codeword{cb\_nodes} and how big the collective buffer should be with \codeword{cb\_buffer\_size}. Table~\ref{table: coll_io_hints_table} summarises the hints just described.

\begin{table}[!htb]
\centering
\ra{1.5}
\caption{Collective I/O hints in ROMIO.}
\newcolumntype{K}{>{\centering\arraybackslash} m{3cm}}
\newcolumntype{V}{>{\centering\arraybackslash} m{5cm}}
\begin{tabular}{KV}
%\begin{tabular}{@{}p{0.3\columnwidth}p{0.6\columnwidth}@{}}
\toprule
\bf \small Hint & \bf \small Description \\
%\multicolumn{1}{c}{\bf \small Hint} & \multicolumn{1}{c}{\bf \small Description} \\
\midrule
\small \codeword{romio\_cb\_write} & \small \codeword{enable} or \codeword{disable} collective writes \\
\small \codeword{romio\_cb\_read} & \small \codeword{enable} or \codeword{disable} collective reads \\
\small \codeword{cb\_buffer\_size} & \small set the collective buffer size [bytes]\\
\small \codeword{cb\_nodes} & \small set the number of aggregator processes\\
%\multicolumn{1}{c}{\small \codeword{cb\_nodes}} & \multicolumn{1}{p{0.6\columnwidth}}{\centering \small set the number of aggregator processes}\\
\bottomrule
\end{tabular}
\label{table: coll_io_hints_table}
\end{table}

Each of these hints has an effect on collective I/O performance. For example, by increasing the number of aggregators there will be a higher number of nodes writing to the parallel file system and thus a higher chance that one of these will experience variable performance due to load imbalance among available I/O servers, with increasing write time variation and associated global synchronisation cost. Furthermore, by increasing the collective buffer size users can reduce the number of two phase I/O rounds and, consequently, the number of global synchronisation events. Bigger collective buffers will also affect the write cost since more I/O servers will be accessed in parallel potentially increasing the aggregated I/O bandwidth.

Besides the hints described in Table~\ref{table: coll_io_hints_table}, there are other hints that do not directly concern collective I/O but affects its performance. The first is the \codeword{striping\_factor} hint, which defines how many I/O targets will be used to store the file. The second is the \codeword{striping\_unit} hint, which defines how big the data chunks written to each I/O target will be (in bytes). These two hints change the file characteristics in the parallel file system and typically the `striping\_unit' also defines the locking granularity for the file (e.g. Lustre).

%\todo[inline]{Add a description of the typical HPC application workflow when using MPI-IO collective write.}
%\subsection{MPI-IO in HPC Applications' Workflow}
%\label{subsec: hpcapp}
%Simplifying, most HPC applications can be divided into multiple phases of computation, in which data is produced, and I/O, in which data is written to persistent storage for post-processing purposes as well as defensive checkpoint-restart. Focusing on the I/O phase and considering the case of applications writing to a shared file, the I/O phase can be divided into the following steps:
%\begin{enumerate}
%\item The MPI Info object is initialised using \codeword{MPI\_Info\_set}: this object is used to pass the user's hints to the ROMIO implementation.
%\item The file is opened using \codeword{MPI\_File\_open}: at this point the info object is passed to the underlying ROMIO layers.
%\item The file view is set using \codeword{MPI\_File\_set\_view}: this describes the data layout in the file for every process and thus contains the access pattern information.
%\item Data is written to the file using \codeword{MPI\_File\_write\_all}: this function invokes the underlying `ADIOI\_GEN\_WriteStridedColl' previously described in Figure~\ref{figure: coll_io_impl}.
%\item The file is closed using \codeword{MPI\_File\_close}: after the file is closed data must be visible to every process in the cluster (MPI-IO does not support the same consistency semantic of POSIX).
%\end{enumerate}

%Figure~\ref{figure: workflow1} shows an example of symplified HPC application workflow with the MPI-IO functions just described. 
%\begin{figure}[!htb]
%  \centering
%  \includegraphics[width=0.5\columnwidth]{figures/workflow1}
%  \caption{}
%  \label{figure: workflow1}
%\end{figure}
%All the collective I/O routines in MPI-IO are blocking, that is, the implementation will return control to the application only when I/O is complete. The MPI-IO specifications also define a set of split collective I/O routines in which non-blocking collective I/O can be started using a `begin' function and completed using an `end' function. Nevertheless, the actual implementation supporting such features is currently missing and the overall application runtime is still highly dependent on I/O.

%!TEX root = ../main.tex
\section{MPI-IO Hints Extensions}
\label{sec: e10-extensions}

To the best of our knowledge, at the time of writing this paper, there is very little or no work on how to use non-volatile memory devices in computing nodes of an HPC cluster as persistent cache layer to boost collective I/O performance in ROMIO. The use of these devices can greatly increase parallelism, reduce write response time variations among processes and consequently global synchronisation cost. Data cached in locally attached SSDs can be synchronised independently by every aggregator in the background while the application can progress doing useful work, effectively converting collective I/O to independent I/O when writing to the parallel file system.

To take advantage of attached non-volatile memories in computing nodes we introduced a new set of MPI-IO hints, reported in Table~\ref{table: hints_table}, and a corresponding set of modifications in the ROMIO implementation of the Universal File System (UFS) ADIO driver supporting them.

\begin{table}[!htb]
\centering
\ra{1.5}
\caption{Proposed MPI-IO hints extensions.}
\newcolumntype{K}{>{\centering\arraybackslash} m{4cm}}
\newcolumntype{V}{>{\centering\arraybackslash} m{4cm}}
\begin{tabular}{KV}
%\begin{tabular}{@{}p{0.55\columnwidth}p{0.43\columnwidth}@{}}
\toprule
\bf \small Hint & \bf \small Value \\
%\multicolumn{1}{c}{\bf \small Hint} & \multicolumn{1}{c}{\bf \small Value} \\
\midrule
\small \codeword{e10\_cache} & \small \codeword{enable}, \codeword{disable}, \codeword{coherent}\\
\small \codeword{e10\_cache\_path} & \small cache directory pathname\\
\small \codeword{e10\_cache\_flush\_flag} & \small \codeword{flush\_immediate}, \codeword{flush\_onclose}\\
\small \codeword{e10\_cache\_discard\_flag} & \small \codeword{enable}, \codeword{disable}\\
\small \codeword{ind\_wr\_buffer\_size} & \small synchronisation buffer size [bytes]\\
%\multicolumn{1}{c}{\small \codeword{ind\_wr\_buffer\_size}} & \multicolumn{1}{p{0.43\columnwidth}}{\centering \small synchronisation buffer size}\\
\hline
\end{tabular}
\label{table: hints_table}
\end{table}

The new hints are used to control the data path in the storage system as well as to define a basic set of cache policies for synchronisation and space management. In particular, the \codeword{e10\_cache} hint is used to \codeword{enable} or \codeword{disable} the cache, directing applications' data to the local file system instead of the global file system. When the hint is set to \codeword{coherent} all the written data extents will be locked until cache synchronisation is completed. The \codeword{e10\_cache\_path} hint is used to control where in the local file system tree the cache file will reside. The \codeword{e10\_cache\_flush\_flag} hint is used to control the synchronisation policy of cached data to the global file. If the hint is set to \codeword{flush\_immediate} data will be immediately flushed to the global file. Alternatively, if the hint is set to \codeword{flush\_onclose} data will be flushed to the global file when it is closed. The \codeword{e10\_cache\_discard\_flag} hint is used to perform basic cache space management. In particular, if the hint is set to \codeword{enable} the cache file will be removed after the file is closed, otherwise (\codeword{disable}) it will be retained until the user manually removes it. Finally, the \codeword{ind\_wr\_buffer\_size} hint controls the size of the buffer used to synchronise cached data to the global file. This hint already existed in ROMIO but was only used during independent I/O to determine the write granularity. The hints in Table~\ref{table: hints_table} can be used in conjunction with the collective I/O hints described in Section~\ref{subsec: hints}.

Besides the proposed cache policies, more complex ones are possible. For example, the cache synchronisation could take into account the level of congestion of the I/O servers. The cache replacement policy could also use a more complex strategy to evict cached files (or extents of data inside the file). Although these can be implemented in ROMIO, they introduce more sophisticated functionalities that go beyond the scope of this work.

\subsection{Cache Hints Integration in ROMIO}
\label{subsec: support}
As already mentioned, the introduced MPI-IO hints are supported by a corresponding set of modifications in the ROMIO implementation~\cite{E10-DEEPER}. These modifications, following described, provide the functionalities necessary to handle the additional cache layer:

\begin{itemize}
        \item \codeword{ADIOI\_Sync\_thread\_start()}: is a new implemented routine providing cache synchronisation in ROMIO. It uses a dedicated POSIX thread to read data back from the cache file into the synchronisation buffer, and write it to the global file;
        \item \codeword{ADIOI\_Cache\_alloc()}: is a new implemented routine providing cache space allocation in ROMIO. It uses the \codeword{fallocate()} system call to efficiently allocate space in the local file system\footnote{For file systems that do not support the fallocate syscall the implementation reverts to standard allocation methods which physically writes zeros to the file, at the cost of time efficiency.};
        \item \codeword{ADIOI\_GEN\_OpenColl()}: is the routine providing collective file open in ROMIO. In the new implementation, when \codeword{e10\_cache} is set to \codeword{enable}, this routine also opens the cache file and stores its MPI file handle in the \codeword{cache\_fd} field, added for the purpose inside the global MPI file handle. If for any reason the open of the cache file fails, the implementation reverts to standard open;
        \item \codeword{ADIOI\_GEN\_WriteContig()}: is the routine providing contiguous file write in ROMIO. In the new implementation, when \codeword{e10\_cache} is set to \codeword{enable}, this routine uses the \codeword{cache\_fd} file handle to write data. Additionally, it creates a synchronisation request (with associated \codeword{MPI\_Request} handle) for the written extent~\cite{mpispecs} and sends it to \codeword{ADIOI\_Sync\_thread\_start()}, which will take care of moving data to the global file system. When data transfer is complete the sync function invokes \codeword{MPI\_Grequest\_complete()} on the \codeword{MPI\_Request} handle associated with the request;
        \item \codeword{ADIO\_Close()}: is the routine providing file close in ROMIO. In the new implementation, when \codeword{e10\_cache} is set to \codeword{enable}, this routine also invokes the \codeword{ADIOI\_GEN\_Flush()} routine to make sure that all the data in the cache has been moved to the global file system, and finally closes the cache file as well as the global file;
        \item \codeword{ADIOI\_GEN\_Flush()}: is the routine providing file flushing in ROMIO. In the new implementation, when \codeword{e10\_cache\_flush\_flag} is set to \codeword{flush\_immediate}, it takes care of checking that previously created synchronisation requests have been completed by invoking \codeword{MPI\_Wait()} on the associated \codeword{MPI\_Request} handle. Alternatively, when \codeword{e10\_cache\_flush\_flag} is set to \codeword{flush\_onclose}, it sends all the pending synchronisation requests to \codeword{ADIOI\_Sync\_thread\_start()} and then waits for them to complete as described above.
\end{itemize}

%In particular, the \codeword{ADIOI\_GEN\_OpenLocal} routine opens the file in the local file system cache to which data will be written. The local pathname is computed as the base name of the global pathname, prefixed with the pathname passed by the user through the \codeword{e10\_cache\_path} hint. Thus, if the file has global \codeword{pathname} = \codeword{/user1/project1/foo} and \codeword{e10\_cache\_path} = \codeword{/tmp/user1/project1}, the local pathname will be \codeword{/tmp/user1/project1/foo}. Of course, local opens can possibly fail for a number of reasons (e.g. the local pathname passed by the user does not exist). In all these cases the implementation will revert to standard collective I/O resetting the \codeword{e10\_cache} hint to \codeword{disable}.
%If the file in the local file system is opened successfully, data is written to it using the \codeword{ADIOI\_GEN\_WriteContigLocal} routine which replaces \codeword{ADIOI\_GEN\_WriteContig} in the diagram in Figure~\ref{figure: coll_io_impl}.

%The \codeword{ADIOI\_GEN\_IfileSync} routine is responsible for the synchronisation of the local cache file to the global file system. Depending on the value of the flush hint it will be invoked immediately after \codeword{MPI\_File\_write\_all} returns from writing all the data to the local cache (\codeword{flush\_immediate}), see Figure~\ref{figure: coll_io}, or at close time (\codeword{flush\_onclose}). The difference is that in the first case the application can progress with computation while the non-blocking synchronisation routine does its job. In the second case, on the other hand, the implementation will busy wait until synchronisation is complete before returning the control to the application. The \codeword{ADIOI\_GEN\_IfileSync} routine is implemented using the MPI Generalised Request interface~\cite{mpispecs}, which allows users to write non-blocking routines as well as call back functions to be used by the MPI implementation to control the state and the progress of an additional user thread through the \codeword{MPI\_Request} object and \codeword{MPI\_Wait} function. 
%In our implementation we use a pthread to synchronise the content of the local file with the parallel file system.%The \codeword{ind\_wr\_buffer\_size} hint, previously mentioned, is used by the synchronisation routine to set the size of the buffer used to read data from the local file and write it to global file, and thus affects the number of synchronisation rounds.

%The \codeword{ADIOI\_GEN\_CloseLocal} routine is used to close the local cache file and start data synchronisation if \codeword{e10\_cache\_flush\_flag} = \codeword{flush\_onclose}. Finally, the \codeword{ADIO\_CacheAlloc} routine is used by aggregators to reserve space in the local cache file system. The routine uses the \codeword{fallocate} system call to allocate space in the file system's data structures instead of writing data to the file, and is therefore more time efficient\footnote{For file systems that do not support the fallocate syscall the implementation reverts to standard allocation methods which physically writes zeros to the file, at the cost of time efficiency.}. If there is not enough space available in the cache, the implementation will revert to standard collective I/O, resetting the \codeword{e10\_cache} to \codeword{disable}.

\subsection{Consistency Semantics}
\label{subsec: consistency}
As far as write consistency is concerned, the MPI-IO interface does not make any assumption regarding the underlying storage system or its semantics. ROMIO specifically supports file systems that are both POSIX compliant, like Lustre, and non-POSIX compliant, like NFS or PVFS. In MPI-IO, written data becomes globally visible only after either \codeword{MPI\_File\_sync()} or \codeword{MPI\_File\_close()} are invoked on the MPI file handle and by default there is no write atomicity. The motivation is that data can be cached at some level locally in the compute nodes. The ROMIO implementation can overcome the risk of data inconsistency, e.g. related to false sharing of file system blocks, using persistent file realms~\cite{ColomaCWWRP04}, and can even enforce atomicity using \codeword{MPI\_File\_set\_atomicity()}.

In our implementation we comply to the MPI-IO semantics just described. This means that data written to the local file system cache using the newly introduced MPI-IO hints will be globally visible to the rest of the nodes only under the following circumstances:
\begin{itemize}
\item The \codeword{e10\_cache\_flush\_flag} has been set to \codeword{flush\_immediate} by the user and synchronisation, started automatically by the implementation right after the write operation, has completed;
\item The \codeword{e10\_cache\_flush\_flag} has been set to \codeword{flush\_onclose} by the user and the invoked \codeword{MPI\_File\_close()} has returned;
\item The \codeword{MPI\_File\_sync()} function has been invoked by the user and it has returned.
\end{itemize}

%At any time users can make sure their data is persistent by invoking \codeword{MPI\_File\_sync} on the MPI file handle. This will call the \codeword{ADIOI\_GEN\_FlushLocal} routine, previously described, that returns only after all the cached data has been synchronised to the global file system or immediately if synchronisation has been already completed. This is perfectly aligned with the MPI-IO consistency semantic which also requires the invocation of \codeword{MPI\_File\_sync} or \codeword{MPI\_File\_close} to ensure that local cached data is persistent in the global file system.

Consistency for reading data from the cache is not clearly defined by the ext2ph algorithm. In general, data written to the local file system cache can be read back from the user without accessing the global file system. Nevertheless, the algorithm calculates the location of a data block based on the number of aggregators, their logical position within the set of aggregators, and the size of the complete data set. This means that a collective read that matches the previous write could safely read the data from the aggregators' cache without incurring any problem. In spite of that, in general reading from the cache requires additional metadata describing the file layout across the caches. For this reason, we currently do not support reads from the local file system cache.

%In general, data written to the local file system cache could be read back from the user without accessing the global file system. Nevertheless, the local cache file in each aggregator only contains the corresponding file domain, which can vary with the number of processes used to run the experiment as well as the number of aggregators selected. This means that a collective read that matches the configuration (number of aggregators) and I/O pattern of the previous write could safely read the data from the aggregators' cache without incurring into any problem. In spite of that, in general reading from the cache requires additional metadata describing the file layout across the caches. For this reason, we currently do not support reads from the local file system cache. %Instead, it is responsibility of the user to make sure that data is persistent in the global file system before reading it back (i.e. by closing the file or flushing the cache content).
%Most HPC applications are simulation codes that have write dominated I/O patterns. They write data to a shared file (or multiple files) for later post processing or defensive checkpoint restart and then progress with computation. Data is typically not read back during normal execution and thus cache coherency is not required. In any case, 

Furthermore, whenever required, we can enforce cache coherency ensuring that read operations cannot access data that is currently in transit, i.e., not or only partially moved from the cache to the global file. This can be done by locking the file domain extent being cached until all the data has been made persistent in the global file. For this purpose ROMIO provides a set of internal locking macros, namely \codeword{ADIOI\_WRITE\_LOCK}, \codeword{ADIOI\_READ\_LOCK} and \codeword{ADIOI\_UNLOCK} that we used in our implementation. The lock of cached data can be selected by setting the \codeword{e10\_cache} hint in Table~\ref{table: hints_table} to \codeword{coherent}. This will \codeword{enable} the cache and set locks appropriately, assuming underlying file system support.

%Typically, if ROMIO can detect stripe size and stripe factor for the file, it will automatically align file domains to the file system stripe, avoiding concurrent locking of the same block by multiple aggregators. In this case, the enforcement of cache coherency through locking will not degrade performance.
%Currently, in order to avoid partially synchronised files, i.e. in the case of a node failing, for new files, the global pathname is hidden at the time of open and made visible again only after close, once all the data has been synchronised and consistency is ensured. If the file already exists, on the other hand, its pathname is not altered. 

\subsection{Changes to the Application's Workflow}
\label{subsec: new-workflow}
Simplifying, most HPC applications can be divided into multiple phases of computation, in which data is produced, and I/O, in which data is written to persistent storage for post-processing purposes as well as defensive checkpoint-restart. Focusing on the I/O phase and considering the case of applications writing to a shared file, the I/O phase can be divided into the following steps:
\begin{enumerate}
        \item The file is opened using \codeword{MPI\_File\_open()}: at this point the info object containing the user defined MPI-IO hints is passed to the underlying ROMIO layers.
        \item Data is written to the file using \codeword{MPI\_File\_write\_all()}: these functions invoke the underlying \codeword{ADIOI\_GEN\_WriteStridedColl()} previously described in Figure~\ref{figure: coll_io_impl}.
        \item The file is closed using \codeword{MPI\_File\_close()}: after the file is closed data must be visible to every process in the cluster. 
\end{enumerate}

To take advantage of the proposed MPI-IO hint extensions, the application's workflow has to be modified. Figure~\ref{figure: workflow3} shows the classical application's workflow (cache disabled) as well as the modified version using the new hints (cache enabled). The difference is that, in order to take advantage of the proposed hints and hide the cache synchronisation to the computation phase, the \codeword{MPI\_File\_close()} for the I/O phase `k' has been moved at the beginning of the I/O phase `k+1', just before the new file is opened.
\begin{figure}[!htb]
  \centering
  \includegraphics[width=\columnwidth]{figures/workflow3}
  \caption{Standard and modified workflows. When cache is disabled compute phase `k+1' starts after file `k' has been closed. When the cache is enabled compute `k+1' can start immediately after data has been written. At the same time, background synchronisation of cached data starts. File `k' is closed before the file `k+1' is opened, forcing the implementation to wait for cache synchronisation to complete.}
  \label{figure: workflow3}
\end{figure}

Since the workflow modification just presented might not be feasible for legacy applications, we developed a MPI-IO wrapper library (called MPIWRAP), written in C++, that can reproduce this change behind the scenes. The library can be linked to the application or preloaded with \codeword{LD\_PRELOAD} and has been used for all the experiments contained in this paper. MPI-IO hints are defined in a configuration file and passed by the library to \codeword{MPI\_File\_open()}. We can define multiple hints targeting different files or groups of files. The library overloads \codeword{MPI\_\{Init,Finalize\}()} and \codeword{MPI\_File\_\{open,close\}()} using the PMPI profiling interface. The workflow modification can be triggered for a specific set of files (identified by the same base name) in the configuration file. Whenever one of such files is closed, our \codeword{MPI\_File\_close()} implementation will return success. Nevertheless, the file will not be really closed. Instead, its handle will be kept internally for future references. When the next shared file with the same base name is opened, our \codeword{MPI\_File\_open()} implementation will search for outstanding opened file handles and will invoke \codeword{PMPI\_File\_close()} on them before opening the new file, thus triggering the cache synchronisation completion check for each of them.

\subsection{I/O Bandwidth}
\label{subsec: bw-impr}
According to the new I/O workflow, described in this section, we have that being $S(k)$ the amount of data written during phase `k', $T_c(k)$ the time needed to write $S(k)$ collectively to the cache using \codeword{MPI\_File\_write\_all()}, $T_s(k)$ the time needed to synchronise the cached data in every aggregator to the global file system (through \codeword{ADIOI\_Sync\_thread\_start()}), and $C(k+1)$ the computation time of phase `k+1', the resulting I/O bandwidth for `k' is expressed by Equation~\ref{formula: bw}:

\begin{equation}\label{formula: bw}
        bw(k) = \frac{S(k)}{T_c(k) + max(0,\ T_s(k) - C(k+1))} \\
\end{equation}
Therefore, the total average bandwidth perceived by the application is:
\begin{equation}\label{formula: abw}
        BW = \frac{\sum_{k=0}^{N-1} S(k)}{\sum_{k=0}^{N-1} T_c(k) + max(0,\ T_s(k) - C(k+1))}
\end{equation}

From Equation~\ref{formula: bw} (in which we have considered the open time neglectable) it is clear that the maximum performance can be obtained when $C(k+1) \geq T_s(k)$, that is, when we can completely hide cache synchronisation by the computation phase. On the other hand when $C(k+1) < T_s(k)$ we might have a minima in the bandwidth since \codeword{MPI\_File\_close()} needs to wait for cache synchronisation completion (Figure~\ref{figure: workflow3}). 
%Finally, being $T_g$ the standard collective write time to the global file system, we assume that $T_g >> T_c$. This assumption is legitimated by the fact that in very large scale systems the number of compute nodes (and thus the number of NVM devices) is orders of magnitude bigger than the available storage targets.

%!TEX root = ../main.tex
\section{Evaluation}
\label{sec: results}
To evaluate the proposed MPI-IO hints we use three popular I/O benchmarks frequently adopted to profile collective I/O performance in other research works: coll\_perf\footnote{Collective I/O benchmark distributed with the MPICH package.}, Flash-IO and IOR.
Minor changes to the source code of the three benchmarks have been made to adapt them to our needs. For example, coll\_perf and Flash-IO do not support writing to multiple files and the emulation of computing delays. Thus, we modified them to reproduce the workflow shown in Figure~\ref{figure: workflow3}. The number of written files and a compute delay are now parameters that can be passed from the command line. In all our tests we used 512 MPI processes distributed over 64 nodes (8 procs/node), fixed the file stripe size to 4~MB and the stripe count to 4. Moreover, for simplicity, we also fixed the size of the cache synchronisation buffer (i.e. \codeword{ind\_wr\_buffer\_size}) to 512~KB, which corresponds to the standard independent I/O buffer size. On the other hand, we varied the collective I/O parameters, i.e., the number of aggregators (from 8 to 64) and the collective buffer size (from 4~MB to 64~MB). For every combination of these parameters (<aggregators>\_<coll\_bufsize>) each benchmark writes four files of the same size (32~GB) with a compute delay of 30 seconds, which is in most cases enough to hide the synchronisation time. We compute the bandwidth as the average bandwidth over the four collective write operations (Equation~\ref{formula: abw}). The different  contributions within the collective I/O write path shown in Figure~\ref{figure: coll_io_impl} are extracted from the ROMIO layer using MPE profiling~\cite{mpe}.
Whenever the compute delay is not enough to hide synchronisation (e.g. when a small number of aggregators is used), the remaining synchronisation time is added to the total write time, thus reducing the bandwidth.

\subsection{Testbed}
\label{subsec: testbed}
Our testbed is a research cluster designed and developed in the context of the DEEP/-ER~\cite{deep}\cite{deep-er} projects (Dynamic Exascale Entry Platform/-Extended Reach). The DEEP/-ER cluster has 2048 cores distributed over 128 computing nodes (dual socket Sandy Bridge architecture). The storage system is composed of 6 Dell PowerEdge R520 servers equipped with 2 Intel Xeon Quad-core CPUs and 32~GB of memory and run the BeeGFS file system from Fraunhofer ITWM~\cite{fhgfs} (formerly known as FhGFS). The servers are connected to a SGI JBOD with 45 2TB SAS drives through a SAS switch using two 4x ports at 6~GB/s, for a total of four 8+2 RAID6 storage targets and 2 RAID1 targets for metadata and management data (1 drive is left as spare). One of the six I/O servers is dedicated as metadata server, one as management server and the remaining four as data I/O servers.
Additionally, every compute node is equipped with 32~GB of RAM memory and a 80~GB SATA SSD containing the operating system plus an additional 30~GB ext4 partition (mounted under `/scratch') for general purpose storage. This partition, in our case, is used to locally cache collective writes. Finally, all the computing nodes are connected through an Infiniband QDR network and use ParaStation MPI~\cite{parastation} (PSMPI) version 5.1.0-1 as message passing library.

\subsection{Coll\_perf}
\label{subsec: coll_perf}
In our coll\_perf configuration every process writes one 64~MB block being part of a tridimensional distributed array to a shared file, thus generating a strided pattern. For every experiment, in which we vary the number of aggregators and the collective buffer size, we measure the coll\_perf perceived write bandwidth in three different cases: \textit{1}) when writing directly to the global file system (BW Cache Disabled), \textit{2}) when writing to the cache (BW Cache Enabled) and afterwards flushing its content to the global file system asynchronously, and \textit{3}) when writing to the cache without flushing its content to the global file system (TBW Cache Enable). The last case provides the theoretical bandwidth achievable when the cache synchronisation cost is completely hidden. Additionally, our coll\_perf experiments do not include the last write phase contribution (Figure~\ref{figure: workflow3}). In fact, for the last write the cache synchronisation cost cannot be hidden since there is no following compute phase. We will show the effect that the last write has on the average bandwidth of the IOR benchmark at the end of this evaluation.

Figure~\ref{figure: collperf-bw} shows the write bandwidth for the three cases previously discussed.
\begin{figure}[b!]
  \centering
  \includegraphics[width=0.95\columnwidth]{figures/coll_perf_32GB_30sec_bw}
  \caption{coll\_perf perceived bandwidth for all combinations of <aggregators>\_<coll\_bufsize>.} % The figure also shows the theoretical bandwidth (TBW) when the cache is not flushed.}
  \label{figure: collperf-bw}
\end{figure}
First of all, we can observe that for most of the experiments the aggregate bandwidth when using the cache is higher than the bandwidth measured when using only the global file system. In particular, we can reach a peak performance of about 20~GB/s, compared to the 2~GB/s of the standard case (BW Cache Disabled), which gives a ten fold improvement. Second, when the number of aggregators is equal to 8, we notice a reduced performance, as the synchronisation cost cannot be completely hidden.

\begin{figure}[b!]
  \centering
  \includegraphics[width=0.95\columnwidth]{figures/coll_perf_32GB_30sec_elapsed_enable}
  \caption{coll\_perf collective I/O contribution breakdown when cache is enabled.} % When the number of aggregators is 8 we clearly see the effect of synchronisation.}
  \label{figure: collperf-elaps-enable}
\end{figure}
The effect of the non-hidden cache synchronisation (not\_hidden\_sync) is shown in Figure~\ref{figure: collperf-elaps-enable}. This figure presents the collective I/O performance breakdown for all the components shown in Figure~\ref{figure: coll_io_impl}. As we can see, although the theoretical bandwidth (TBW Cache Enable) peaks at 4~GB/s (Figure~\ref{figure: collperf-bw}), the measured bandwidth (BW Cache Enable) can be even worse than the standard case (BW Cache Disable). This happens because the cache data cannot be flushed to the global file system quickly enough. In all the other experiments, the perceived and theoretical bandwidth are well aligned.

\begin{figure}[htb]
  \centering
  \includegraphics[width=0.95\columnwidth]{figures/coll_perf_32GB_30sec_elapsed_disable}
  \caption{coll\_perf collective I/O contribution breakdown when cache is disabled.}
  \label{figure: collperf-elaps-disable}
\end{figure}

As already said in the previous sections, our SSDs based approach can also help to reduce the global synchronisation cost in the extended two phase I/O algorithm at the base of collective I/O.
In fact, by comparing Figures~\ref{figure: collperf-elaps-enable} and~\ref{figure: collperf-elaps-disable}, we see that the global synchronisation costs in collective I/O, represented by \codeword{MPI\_Alltoall()} (shuffle\_all2all) and \codeword{MPI\_Allreduce()} (post\_write) are consistently reduced when using the cache.

Finally, we observe that most of the times, when using the cache, larger collective buffers do not produce large performance improvements. This means that we can achieve good performance with small buffers and thus reduce the memory pressure of collective write operations on compute nodes.

\subsection{Flash-IO}
\label{subsec: flash}
Flash-IO is the I/O kernel of the Flash application. Flash is a block-structured adaptive mesh hydrodynamics code. The computational domain is divided into blocks which are distributed across the processors. Typically a block contains 16 zones in each coordinate direction (x,y,z) and a perimeter of guard cells (presently 4 zones deep) to hold information from the neighbors. The application writes three files using the parallel HDF5 library: a checkpoint file, a plot file without corner data and a plot file with corner data. The checkpoint file is the biggest of the three and consumes the majority of the I/O time. In our configuration the checkpoint file contains 80 blocks/process and each of the 16 zones/block contains 24 variables encoded with 8bytes (768~KB/proc/block). Therefore, the total size is slightly bigger than 30~GB (including metadata).

\begin{figure}[t!]
  \centering
  \includegraphics[width=0.95\columnwidth]{figures/flash_32GB_30sec_bw}
  \caption{Flash-IO perceived bandwidth for all combinations of <aggregators>\_<coll\_bufsize>.} %The figure also shows the theoretical bandwidth (TBW) when the cache is not flushed.}
  \label{figure: flash-bw}
\end{figure}

Figure~\ref{figure: flash-bw} shows the write bandwidth perceived by Flash-IO for all the experiments performed in the different cases under study. Similarly to coll\_perf, when the cache is enabled we can hide the cache synchronisation cost for most of the experiments. Once again, like in the previous case, when the number of aggregators is equal to 8 we have a mismatch between the perceived and the theoretical bandwidth. For Flash-IO the peak bandwidth is about 40~GB/s when using 64 aggregators and 4~MB collective buffer size, against the 2~GB/s measured when writing directly to the parallel file system.
\begin{figure}[htb]
  \centering
  \includegraphics[width=0.95\columnwidth]{figures/flash_32GB_30sec_elapsed_enable}
  \caption{Flash-IO collective I/O contribution breakdown when cache is enabled.}
  \label{figure: flash-elaps-enable}
\end{figure}

Figure~\ref{figure: flash-elaps-enable} shows the effect of cache usage on the different collective I/O performance contributions. We can clearly see that when the number of aggregators is equal to 8 the cache synchronisation cannot be completely hidden, causing the bandwidth mismatch previously observed in Figure~\ref{figure: flash-bw}. Furthermore, like in the coll\_perf case the global synchronisation contributions can be reduced when using the cache, and so can the memory pressure on the compute nodes.
\begin{figure}[htb]
  \centering
  \includegraphics[width=0.95\columnwidth]{figures/ior_32GB_30sec_bw}
  \caption{IOR perceived bandwidth for all combinations of <aggregators>\_<coll\_bufsize>.} %The figure also shows the theoretical bandwidth (TBW) when the cache is not flushed.}
  \label{figure: ior-bw}
\end{figure}
Nevertheless, for the 64 aggregators and 16~MB collective buffer size configuration the global synchronisation overhead associated to \codeword{MPI\_Allreduce()} (post\_write) has a larger value. The outlier causes a strong reduction in the write bandwidth, although we can still achieve more than 10~GB/s. This indicates that the effect of global synchronisation when using the cache can be even more severe, due to the much higher bandwidth achievable compared to the standard global file system approach.

\subsection{IOR}
\label{subsec: ior}
We tested IOR when writing collectively to a shared file. Each of the 512 MPI processes writes one 8~MB block of data for each of the 8 segments, that is, a 32~GB file during every test.
%\todo[inline]{remove standard deviation explanation and focus on the last part where the real reason for reduced performance is exposed. explain that add one additional read and one additional write is making things worse when using a small number of SSD/aggregators. improve figures adding bars for synchronization cost separately. add a reference to the statement that large systems might not be able to write concurrently to all the available I/O targets.}
%\todo[inline]{reading data back from the cache: illustrate problems related to additional metadata information required (example ADIOS does this using a custom binary format). describe this in related or future work.}
As for the previous two benchmarks, Figure~\ref{figure: ior-bw} shows the write bandwidth perceived by IOR. Nevertheless, unlike the previous benchmarks, in IOR we also take into account the non-hidden synchronisation cost coming from the last write phase. In this case, the peak bandwidth when using the cache can reach about 6~GB/s versus the 2~GB/s of the standard collective write to the global file system. We can also see that the theoretical bandwidth is aligned with the figures presented for coll\_perf and Flash-IO. In fact, although we can hide cache synchronisation costs for the three intermediate write phases in IOR, the fourth write phase will limit the peak performance.

Figures~\ref{figure: ior-elaps-enable} shows the collective I/O cost breakdown for all the time contributions. 
\begin{figure}[htb]
  \centering
  \includegraphics[width=0.95\columnwidth]{figures/ior_32GB_30sec_enable}
  \caption{IOR collective I/O contribution breakdown when cache is enabled.}
  \label{figure: ior-elaps-enable}
\end{figure}
In this figure we can clearly observe the not\_hidden\_sync term preventing IOR from achieving higher performance. This is added to the other time contributions and corresponds to the term $T_s(k)-C(k+1)$ reported in Equation~\ref{formula: bw}. In our specific case, $k = 4$ and $C(4+1) = 0,$ meaning that the total write time is accounting for the whole $T_s(4)$ term.


%%!TEX root = ../main.tex
\section{Evaluation}
\label{sec: evaluation}
We now present the evaluation of our \textit{Assisted I/O library} and \textit{Advice Manager} prototypes with a real world application used by physicists at the data processing center of the University of Mainz (ZDV). Our testbed is composed by %two separate systems: 
a test cluster of seven nodes, mainly intended to evaluate the proposed Linux kernel modification with the Lustre file system. %, and the Mogon cluster, currently the production system at the ZDV. 
We start with a concise description of the %two 
system as well as a detailed analysis of the target application's I/O pattern, and then present the results of our experiments. 

We evaluate the performance of our framework using two metrics, the execution time of the test application and the number of reads completed by every target file system. %To simulate a heavily loaded cluster we use a background process that runs on an set of nodes independent from the node running the target application. Additionally, we also measure the overhead introduced by our prototype. 

\subsection{Test Cluster}
\label{subsec: test_cluster}
%As already mentioned, this small cluster is aimed mainly to test our modified kernel with Lustre. The reason is that it was not possible to disrupt the production cluster, affecting hundreds of users, by re-installing the operating system kernel. 
%In order to make realistic comparisons between Lustre and GPFS, the test cluster also mounts a GPFS file system. The only GPFS network shared disk (NSD) servers and Lustre object storage servers (OSS) are hosted by two machines (DELL R710 equipped with two quadcore E5620 @ 2.4GHz and 24GB main memory) of the seven available. The disk setup consists of one DELL MD3200 and four MD1200, connected to the two storage servers in a failover configuration. Each of the storage enclosures is equipped with 12 disks (for a total of 60 disks) and configured with 8+2 RAID 6 (6 8+2 RAIDs in total). Four of the six available RAIDs are used as Lustre OSTs and GPFS NSDs (with separated partitions). The Lustre MDS is hosted by a SuperMicro Chassis equipped with one quadcore Xeon E3-1230 @ 3.3GHz and 16GB of main memory. In this case the metadata target (MDT) is a 120 GB SSD Intel 520. The remaining four machines, equipped with an eight core E3-1230 @ 3.3GHz processor and 24 GB of main memory, work as compute nodes and file system clients. All of the machines are connected to a 24 ports SSE-245 Gbit ethernet switch. Both the GPFS and Lustre file systems are formatted with a block size of 4MB.

%Markus description
As already mentioned, this small cluster is aimed mainly to test our modified kernel with Lustre. 
The reason is that it was not possible to disrupt the production cluster, affecting hundreds of users, by re-installing the operating system kernel.
In order to make realistic comparisons between Lustre and GPFS, the test cluster also has a GPFS file system on comparable hardware. 
Both filesystems have a single disk server each, one Dell R710 acts as GPFS network shared disk (NSD) server and another as Lustre object storage server (OSS). The R710 are equipped with two quadcore E5620 @ 2.4GHz and 24GB main memory. For storage, both disk servers share a MD3200 array with 2 controllers and 4 MD1200 expansion shelves for a total of 60 2TB drives. The Storage is formatted in 4 15 dynamic disk pools. This is the LSI/Netapp type of declustered RAID, which distributes the 8+2 RAID6 stripes evenly over all 15 disks for better rebuild performance. The disk block size is set to 128KiB, which results in a RAID stripe size of 1MiB. The four disk pools are then split on the Array into LUNs, one of the LUNs from each disk pool is then used for GPFS and another one from each pool is used for Lustre. This results in comparable resources for both filesystems and tests do not interfere with each other, as long as only one filesystem is tested at a time. While the GPFS filesystem embeds the metadata with the data, Lustre needs a separate Metadata Server (MDS). This is hosted by a SuperMicro server equipped with one quadcore Xeon E3-1230 @3.3GHz and 16GB of main memory, as metadata target (MDT) it uses a 120GB SSD Intel 520. Four other machines of the same type, equipped with an eight core E3-1230 @3.3GHz processor and 16GB of main memory, work as compute nodes and file system clients. All machines, servers and clients, are equipped with Intel X520DA 10Gigabit adapters and connected to a SuperMicro SSE-X24S 24 ports 10 Gigabit switch. Both, the GPFS and Lustre file systems are formatted with a block size of 4MB.
%The Lustre file system was formatted with a block size of 4MB. It is composed of 4 OSTs and a MDT.%; the input files we used were belonging each one to a different OSTs. %We also tried to stripe the files over all the OSTs, but didn't notice any difference in the execution times.

%GPFS was also formatted with a block size of 4MB and has 4 NSDs.
 
%\subsection{Production Cluster}
%\%label{subsec: mogon}
%The Mogon cluster has 535 nodes each equipped with four 16 core CPUs @ 2.1GHz processors for a total of 34,240 cores. All of the nodes run the Scientific Linux distribution with kernel version `2.6.32-358.2.1.el6.x86\_64' and are connected via QDR Infiniband network. The native parallel file system is GPFS, formatted with a block size of 4MB like the one in the test cluster. Mogon has eleven GPFS NSD servers serving three separate file systems. We chose one of them to conduct our experiments. We reserved a full node (i.e. 64 cores) via the LSF batch system, specifying 2GB of memory for each core.

%The connection between the nodes is Infiniband. %that if necessary can fall back to RDMA if something is wrong.
% TODO: what sort of Infiniband? Probably 'QDR'?
%GPFS is the native parallel file system in Mogon and therefore we were able to run our advice infrastructure prototype already installed in the production cluster so we were able to test our framework on Mogon. Each node provides 1.5$\,$TB of local disk space, which gave us the possibility to test the ext4 file system in one of these nodes.

\subsection{Real World Application}
\label{subsec: application}
Our target real world application is written using `ROOT', an object-oriented framework widely adopted in the experimental high energy physics community to build software for data analysis. The application analyzes data read from an input file in the `ROOT' format (structured file format). %The file we used is 5GB in size.

\begin{figure}[!htb]
  \centering
  \begin{subfigure}[t]{\columnwidth}
    \centering
    \includegraphics[width=\columnwidth]{advice_paper/figures/iopat_profile}
    \caption{\textit{}}
    \label{figure: iopat_profile}
  \end{subfigure}
  \begin{subfigure}[t]{\columnwidth}
    \centering
    \includegraphics[width=\columnwidth]{advice_paper/figures/00050_zoom}
    \caption{\textit{}}
    \label{figure: iopat_zoom}
  \end{subfigure}
  \caption{I/O read profile of the target application under analysis (\ref{figure: iopat_profile}), extracted from the the GPFS file system in the test cluster, and zoomed window (\ref{figure: iopat_zoom}) showing the actual pattern details.}
  \label{figure: iopattern_with_statistics}
\end{figure}

First of all we characterized the application's I/O pattern for a target file using traces and statistics extracted through several tools such as \textit{strace}, \textit{ioapps}~\cite{ioapps} and GPFS's \textit{mmpmon}~\cite{mmpmon} monitoring tool. Figure~\ref{figure: iopattern_with_statistics} shows the I/O pattern along with some additional statistics. As it can be seen, in this specific case (5GB file), the application issues a total of 10515 \texttt{read()} system calls to read about 2.6GB of total data. The average request size is 250kiB and the time spent waiting for I/O is 12 seconds, when running on the test cluster. 

At a first glance the general I/O behaviour of the application looks linear, most of the accesses to the file follow an increasing offset. Nevertheless, adjacent reads are separated by gaps (a strided read pattern). In a few cases this gap becomes negative, meaning that the application is moving backwards in the file to read some data previously left behind (as reported in Figure~\ref{figure: iopat_zoom}). %(Figure~\ref{figure: 00050_zoom}). 

%\begin{figure}[!htb]
%  \centering
%  \includegraphics[width=0.44\textwidth]{figures/00050_zoom}
%  \caption{Zoomed version of the I/O pattern under analysis. Here backwards seeks are clearly visible.}
%  \label{figure: 00050_zoom}
%\end{figure}

After a detailed I/O pattern analysis we could divide the target file into contiguous non overlapping ranges. Within these ranges reads happen to have increasing offset. %The information extracted was then used to tailor a configuration file of the form described in Listing~\ref{config}, with each discovered range forming a part of the `WillNeed' section. 
Even though the general I/O pattern of the application for different files looks similar\footnote{Due to space limits we do not report the comparison between different files.}, the size of the non overlapping ranges may change significantly. This general behaviour can be modelled using a configuration file in which a `WillNeed' hint covers the whole file from beginning to end (i.e. `Offset' and `Length' equal to 0). The backwards seeks can be accounted for using the `CacheSize' parameter to keep previously accessed blocks in cache. In this way we effectively emulate a sliding window that tracks the application's I/O behaviour. This would not be possible by just using a, e.g., \texttt{POSIX\_FADV\_WILLNEED} advice on the whole file before starting the application like shown by Figure~\ref{figure: fadvise_comparison}. The reason is that if the file size is equal or smaller than the cache size, we would have a large number of valuable pages discarded from the cache to load data that will be accessed at the end of the application. Additionally, if the file size is bigger than the cache size we would have the file system discarding blocks at the beginning of the file as the blocks at the end are preloaded, effectively forcing the application to access these blocks from the I/O servers instead of the cache. With our approach, on the other hand, we keep in the cache only a small, controlled number of blocks (the ones currently accessed), while the older blocks are discarded since no longer needed. %For GPFS this causes an over-specialized configuration file to be generated, that works in one case less effectively than in another. The reason is that GPFS requires the user to release all the hints that have been previously acquired. If the \textit{Advisor Threads} uses a configuration file that does not match the current I/O pattern, prefetched blocks may be released and afterwards accessed again by the application causing a cache miss. For POSIX advice this is not an issue as the API does not require the releasing of prefetched ranges\footnote{This choice was imposed by the fact that \texttt{POSIX\_FADV\_DONTNEED} can be quite time consuming and its employment can considerably slow down the `Advisor Thread', making it useless.}.
% and leave instead the duty to the LRU algorithm in the page cache. In any case, the configuration file mechanism is flexible and users with specific requirements can easily add support for their I/O patterns, improving the performance of their applications.

\begin{figure}[!htb]
  \centering
  \includegraphics[width=\columnwidth]{advice_paper/figures/SC2015/ROOT/separate_plots/test_cluster/test_fadvise_no_border}
  \caption{Comparison between different usage stategies of posix\_fadvise for an input file of 55GB residing in an ext4 file system. The first bar represents the case in which no advice is used, the second bar represents the case in which a POSIX\_FADV\_WILLNEED is issued for the whole file at the beginning of the application and the third bar represents the case in which POSIX\_FADV\_WILLNEED is issued using MERCURY.}
  \label{figure: fadvise_comparison}
\end{figure} 
 
To assess the impact of our prototype on the application and file systems performance we considered the application execution time and the number of reads accounted for by the respective file systems. We conducted our experiments without file system hints and then with file system hints issued transparently to the application by the \textit{Advice Manager}. Furthermore, we ran each experiment three times and calculated average, minima and maxima for each metric. In order to avoid caching affecting our measurements, extra care was taken to clean all the relevant caches for the different file systems. For ext4 and Lustre this was accomplished by using the command line: $$echo\ 3 > /proc/sys/vm/drop\_caches$$ on the file system clients. Additionally, for Lustre this command was also executed on the OSS to avoid the server side cache to be retained. In the case of GPFS, the file system client's page pool was cleaned using the clean file cache hint in Table~\ref{table: hints_table}, the NSD servers do not cache any data. 
% unmounting the file system and remounting it again. %TODO: is this quoting a GPFS manual?
%On the Mogon cluster we could not unmount and shutdown GPFS, so we cleaned the client page pool as mentioned before. We checked using the test cluster that the effect of the missing steps (shutting down GPFS and unmounting it) did not have any effect on our results.

\subsection{Execution Time}
\label{subsec: results}
To measure the performance improvements that our prototype can deliver to the application's runtime we conducted two set of tests. In the first test we varied the size of the input file from 5 to 95GB. This is mainly aimed to study the behaviour of the `ROOT' application using different input file sizes and how our solution behaves when the file becomes bigger than the available cache space. In the second test we varied the number of `ROOT' instances running simultaneously from 1 to 8. By doing so we study the interaction of multiple processes accessing the file system and how these can benefit from the prefetching hints generated by MERCURY. Figure~\ref{figure: runtime} reports the results for the described experiments. All the tests where performed using a `BlockSize' of 4MB, a `CacheSize' of 8 blocks, a `ReadAheadSize' of 4 blocks, and a `WillNeed' hint covering the whole file (i.e. with `Offset' and `Length' equal to 0), resulting in each process consuming up to 32MB of cache space and 512MB in total for 8 application instances. The `WillNeed' on the whole file causes the \textit{Advisor Thread} to issue up to 4 (`ReadAheadSize') prefetching requests for blocks of 4MB sequentially, starting from the current accessed block. This has the same effect of data sieving in ROMIO, optimizing the access size and allowing the application to read the requested data randomly from the cache instead of the file system. The produced effect is particularly beneficial in the case of Lustre and ext4, as it can be seen in Figures~\ref{figure: ext4_1} and~\ref{figure: lustre_1}. In these cases we measure reductions in the execution time of up to 50\% circa, with respect to the normal case. For GPFS we can still observe an improvement but this is more contained compared to the other file systems (Figure~\ref{figure: gpfs_1}). The reductions in the execution time measured in GPFS are on average up to 10\%, with respect to the normal case. The reason is that the default prefetching strategy in GPFS works better that traditional read-ahead. In fact, by disabling the prefetching in GPFS we observed reductions in the execution time comparable to the other file systems (not reported here).
\begin{figure*}[!htb]
  \centering
  \begin{subfigure}[t]{0.32\textwidth}
    \centering
    \includegraphics[width=\textwidth]{advice_paper/figures/SC2015/ROOT/separate_plots/test_cluster/ext4/runtime}
    \caption{\textit{}}
    \label{figure: ext4_1}
  \end{subfigure}
  \begin{subfigure}[t]{0.32\textwidth}
    \centering
    \includegraphics[width=\textwidth]{advice_paper/figures/SC2015/ROOT/separate_plots/test_cluster/gpfs/runtime}
    \caption{\textit{}}
    \label{figure: gpfs_1}
  \end{subfigure}
  \begin{subfigure}[t]{0.32\textwidth}
    \centering
    \includegraphics[width=\textwidth]{advice_paper/figures/SC2015/ROOT/separate_plots/test_cluster/Lustre/runtime}
    \caption{\textit{}}
    \label{figure: lustre_1}
  \end{subfigure}
  \begin{subfigure}[b]{0.32\textwidth}
    \centering
    \includegraphics[width=\textwidth]{advice_paper/figures/SC2015/ROOT/cluster/multiple_instances/simult_instance_ext4_test_cluster}
    \caption{\textit{}}
    \label{figure: ext4_2}
  \end{subfigure}
  \begin{subfigure}[b]{0.32\textwidth}
    \centering
    \includegraphics[width=\textwidth]{advice_paper/figures/SC2015/ROOT/cluster/multiple_instances/simult_instance_gpfs_test_cluster}
    \caption{\textit{}}
    \label{figure: gpfs_2}
  \end{subfigure}
  \begin{subfigure}[b]{0.32\textwidth}
    \centering
    \includegraphics[width=\textwidth]{advice_paper/figures/SC2015/ROOT/cluster/multiple_instances/multiple_simult_procs_Lustre_testcluster}
    \caption{\textit{}}
    \label{figure: lustre_2}
  \end{subfigure}
  \caption{Running time of the ROOT application for the three file system under study using different input file sizes (\ref{figure: ext4_1},~\ref{figure: gpfs_1} and~\ref{figure: lustre_1}) and different number of instances accessing a file of 5GB (\ref{figure: ext4_2},~\ref{figure: gpfs_2} and~\ref{figure: lustre_2}).}
  \label{figure: runtime}
\end{figure*}
\begin{figure*}[!htb]
  \centering
  \begin{subfigure}[t]{0.32\textwidth}
    \centering
    \includegraphics[width=\textwidth]{advice_paper/figures/SC2015/ROOT/separate_plots/test_cluster/ext4/reads}
    \caption{\textit{}}
    \label{figure: ext4_3}
  \end{subfigure}
  \begin{subfigure}[t]{0.32\textwidth}
    \centering
    \includegraphics[width=\textwidth]{advice_paper/figures/SC2015/ROOT/separate_plots/test_cluster/gpfs/server_reads}
    \caption{\textit{}}
    \label{figure: gpfs_3}
  \end{subfigure}
  \begin{subfigure}[t]{0.32\textwidth}
    \centering
    \includegraphics[width=\textwidth]{advice_paper/figures/SC2015/ROOT/separate_plots/test_cluster/Lustre/server_reads}
    \caption{\textit{}}
    \label{figure: lustre_3}
  \end{subfigure}
  \begin{subfigure}[b]{0.32\textwidth}
    \centering
    \includegraphics[width=\textwidth]{advice_paper/figures/SC2015/ROOT/cluster/multiple_instances/reads_simult_instance_ext4_test_cluster}
    \caption{\textit{}}
    \label{figure: ext4_4}
  \end{subfigure}
  \begin{subfigure}[b]{0.32\textwidth}
    \centering
    \includegraphics[width=\textwidth]{advice_paper/figures/SC2015/ROOT/cluster/multiple_instances/reads_simult_instance_gpfs_test_cluster}
    \caption{\textit{}}
    \label{figure: gpfs_4}
  \end{subfigure}
  \begin{subfigure}[b]{0.32\textwidth}
    \centering
    \includegraphics[width=\textwidth]{advice_paper/figures/SC2015/ROOT/cluster/multiple_instances/reads_multiple_simult_procs_Lustre_testcluster}
    \caption{\textit{}}
    \label{figure: lustre_4}
  \end{subfigure}
  \caption{Reads processed by local ext4, GPFS and Lustre I/O servers for various input file sizes (\ref{figure: ext4_3},~\ref{figure: gpfs_3} and~\ref{figure: lustre_3}) and multiple instances of ROOT accessing a file of 5GB (\ref{figure: ext4_4},~\ref{figure: gpfs_4} and~\ref{figure: lustre_4}).}
  \label{figure: read}
\end{figure*}

As far as Figures~\ref{figure: ext4_2},~\ref{figure: gpfs_2} and~\ref{figure: lustre_2} are concerned, these account for the effect of processes' concurrency on the file system. Before continuing with the discussion we have to make a note here. In our architecture, only one process per file system's client issues (through multiple \textit{Advisor Thread}s) hints on behalf of running applications. This introduces some overhead, since we have to pass the access information from the \textit{Assisted I/O library} to the \textit{Advice Manager}, but has the advantage of better coordinating accesses to the same file from multiple processes. Nevertheless, we found that in the case of GPFS, despite the fact of having multiple \textit{Advisor Thread}s, only one process among the many was receiving a benefit from the prefetching hints. The reason is that GPFS seems to have the restriction of hinting only one file per process. For this reason, we developed another variant of MERCURY in which the AIO library, now renamed \textit{Self Assisted I/O library} (SAIO), internally provides the creation and the handling of multiple \textit{Advisor Thread}s. Looking at the figures generated with the new SAIO library we can assess the effectiveness of the prefetching hints for the three file systems considered. In particular, Lustre provides the best runtime improvements compared to the case in which no hints were used. GPFS shows a more contained improvement since the I/O time is already small compared to Lustre and ext4. Finally, ext4 can really benefit from prefetching hints especially for high process counts. Overall, excluding ext4, when we increase the number of processes the runtime improvements shrink. This is probably due to the saturation of the file system client bandwidth.

%Figure~\ref{figure: exec_time_comparison} shows the execution time of the target application in both clusters. As already mentioned, for the experiments we tailored a configuration file in order to fit the target I/O pattern. Additionally, we also used a configuration file containing only a `WillNeed' section covering the whole file. In this last case the \textit{Advisor Thread} moves from the beginning towards the end of the file, prefetching \texttt{GPFS\_MAX\_RANGE\_COUNT} blocks at a time, following the application I/O profile (sliding window prefetch). This is intended to evaluate the relationship between costs and benefits when building a complex configuration file, instead of using a simple one that describes the general I/O behaviour of the application. %We show that a perfectly tailored config file can give better performance than a simple one. On the other hand the complexity and the effort required to build it increase.   

%\begin{figure}[!htb]
%  \centering
%  \includegraphics[width=0.44\textwidth]{figures/exec_time_comparison}
%  \caption{Execution time of the target application on the test cluster and on the `Mogon' cluster for the different available file systems.}
%  \label{figure: exec_time_comparison}
%\end{figure}

%As we can see, when the \textit{Advice Manager} is used to generate the proper hints the execution time can always be reduced. In the best case, on the test cluster this reduction is 44\% (60 seconds) for Lustre, 9\% (8 seconds) for GPFS and 17\% (19 seconds) for ext4. The big difference between Lustre and the other file systems is due to Lustre performing very poorly with the specific type of I/O pattern used (small random reads). As a result, the impact of I/O on the total time, as well as the corresponding reduction, are large compared to the other two cases. For `Mogon', we can save up to 6\% (14 seconds) of the execution time on GPFS and up to 9\% (22 seconds) on ext4. This reduction is particularly significant since in a production environment resources are highly valuable. With our approach we can reduce the application requirement of system resources (I/O and CPU time) making them available to other applications. 

%Finally, in the test cluster we can observe that using a single `WillNeed' section covering the whole file on GPFS performs equally as the tailored configuration file. In comparison ext4 performs better with the simple configuration file, while for Lustre there is no significant difference between the two cases. 
%The reason, as already mentioned, is that if the configuration file is not tailored properly GPFS may release blocks of the file that will be accessed later by the application causing cache misses and therefore degrading the performance. 
%In the `Mogon' cluster there is no visible difference in the performance of these two test cases for GPFS.
%, because the tests were not performed at the same time and the load level of the file system may have changed significantly in the meanwhile. 
%Also in this case, ext4 on Mogon gives its best performance for the configuration covering the whole file, 4\% reduction (10 seconds) compared to the tailored configuration. 

%In conclusion, here we showed that for the specific I/O pattern exposed by the target application, a perfectly tailored configuration file does not necessarily give the best performance in terms of execution time of the application. On the other hand, a configuration file that covers the whole file, capturing the general behaviour of the application can still give significant improvements, with the additional benefit that it is general enough to serve any input file and requires no time to be built.   
%The results shown in Figure~\ref{figure: fullnode_mogon_exectime} are more relevant, as they were obtained on a production cluster. In fact, saving a few seconds from an applications runtime has a large impact because it means that we use less CPU time accross the cluster, allowing more users to run their applications and increasing job throughput for the administrator.
%In Figure~\ref{figure: fullnode_mogon_exectime} we clearly see that on Mogon there is a wide variation in the execution times for the local file system when we run with and without the \textit{Advice Manager} and this effect doesn't not change for the different configurations tested. To be more specific, the improvement in the runtime is 8.6$\%$ and 12$\%$ respectively for the local file system and for GPFS.

\subsection{Read Request Rate}
\label{subsec: reads}
Figure~\ref{figure: ext4_3},~\ref{figure: gpfs_3} and~\ref{figure: lustre_3} report the number of read requests accounted for by the different file systems under study. In the specific, the figures show how the number of reads at the I/O server side for both GPFS and Lustre can be substantially reduced with our approach. This has a significant impact in HPC cluster in which the file system may be accessed by many thousand of processes at the same time. Reducing the number of requests for an application can increase the number of IOPS available for others. This result is also confirmed for multiple instances of the `ROOT' application running concurrently (Figure~\ref{figure: ext4_4},~\ref{figure: gpfs_4} and~\ref{figure: lustre_4}).
%We now report the effect that hints have on the number of reads. To measure this we used three different tools, according to the file system under test. For ext4 we monitored the individual disk involved, acquiring the statistics from the `sysfs' file system. For Lustre we used the \textit{collectl}~\cite{collectl} tool which permits us to measure the reads processed by the OSTs in a simple way. Finally, for GPFS we used the \textit{mmpmon} tool mentioned before, which measures the reads on the file system client side.
%TODO reference for collectl? http://collectl.sourceforge.net/ is there an introductory paper for it?

%Figure~\ref{figure: reads_final_comparison} shows the obtained results in both clusters. As we can see, in the test cluster there is a consistent improvement when the \textit{Advice Manager} is used. For ext4 and Lustre we observe a reduction of circa 50\% in the number of  read requests, as many of the I/O requests are satisfied by the relevant caches populated by the \textit{Advice Manager}. For GPFS the number of read requests processed by the NSD server was reduced by up to 84\% (from 4697 to 762).

%\begin{figure}[!htb]
%  \centering
%  \includegraphics[width=0.44\textwidth]{figures/reads_final_comparison}
%  \caption{Number of read operations accounted for by the different file systems on the test cluster and on the `Mogon' cluster.}
%  \label{figure: reads_final_comparison}
%\end{figure}

%As for the execution time, in this case we can notice that there is a small difference between the number of reads obtained with the tailored configuration file and the configuration file covering the whole file. In particular, the last configuration file performs better than the first one since in this case the implementation prefetches more data.
%The reason for this is that while the last one covers the whole file, the first leaves some small portion of the file uncovered. This was a choice we made to simplify the configuration file. 

%TODO: so its incomplete - you turn the read ahead system off part way through the test?

%The presented results are particularly relevant in the case of highly parallel environments where many applications are using the file system at the same time. Indeed, in this case by reducing the number of requests that the file system has to handle for every file, we can increase the number of IOPS available for other applications and therefore improve the overall performance. 

%\subsection{Effect of Background Processes}
%\label{subsec: bkg_procs}
%One interesting thing to look at is the effect of other processes and applications using resources on the file servers. In this respect, it only make sense to investigate such a scenario for distributed and parallel file systems. For this reason, we show results for Lustre and GPFS on the test clusters. Concerning Mogon, the results for the execution time and the number of read requests processed already include the effect of many processes requesting resources from the file servers. Although we reserved a complete node with 64 cores, there are many other nodes running a multitude of different applications that are putting load on the file servers. When looking at the results it should be no surprise that the execution times of the application running on a clean test environment and on a production cluster differ.

%Table~\ref{table: runtime_bkg} reports the results for execution time and number of reads under heavy load of the file systems by another 60 processes. These processes run on the three free nodes of the test cluster (those not running the application) and each of them continuously generates random reads for 20 files, (for a total of 60 files). In this case we can notice that practically there is no difference between a tailored configuration file and a configuration file covering the whole file, aligning with the results showed in Figure~\ref{figure: exec_time_comparison} for `Mogon' and test cluster. We can also notice that for GPFS the number of read operations does not change. The reason is that \textit{mmpmon} only measures statistics on the file system client side. For Lustre, on the other hand, we also see the contribution of the other processes accessing the file system since we measure the reads at the OSTs.  

%\begin{table*}[!ht]
%\caption{Test cluster's results under heavy loaded file systems.}
%\centering
%\resizebox{1\textwidth}{!}{\begin{minipage}{\textwidth}
%\begin{tabular}{  l  c  c  c  c }
%\toprule
% & \multicolumn{2}{c}{Lustre} & \multicolumn{2}{c}{GPFS} \\
%   & Exec time [sec] & \# reads & Exec time [sec] & \# reads \\
%   \midrule
%   w/o AM & 148.85$\pm$1.47 & 1615623$\pm$11111 & 104.35$\pm$1.48 & 4675$\pm$30 \\ 
%   w/ AM Tailored config & 81.82$\pm$1.17 & 1324290$\pm$4845 & 81.38$\pm$0.46 & 1089 \\ 
%   w/ AM WillNeed all file & 80.48$\pm$0.47 & 1320422$\pm$3476 & 79.72$\pm$0.41 & 762 \\ 
%\bottomrule  
%\end{tabular}
%\end{minipage}}
%\label{table: runtime_bkg}
%\end{table*} 

%\subsection{Overhead}
%\label{subsec: overhead}
%As a final test, we evaluated the overhead introduced by our advice infrastructure prototype to the application. This was done by creating a configuration file that contains no advice for the input file. Correspondingly, the \textit{Interposing I/O Library} intercepts the read calls of the application and forwards them to the \textit{Advice Manager}, but this time it does not generate any hints`. The result is that the application runs normally with the additional overhead of the interprocess communication between \textit{Interposing I/O Library} and \textit{Advice Manager}. With this experiment we observed 0.68\% overhead on the test cluster for Lustre and 3\% for GPFS. Since the communication overhead is constant, the impact on GPFS is bigger being the execution time on GPFS smaller than Lustre. 
%So here we report our findings on the test cluster for Lustre and GPFS. What we do in this case is to try to put load on the respective servers from a number of processes running on three independent nodes, (not those running the application). In each of these nodes we start in parallel 30 instances of a program which performs random reads from specified files on the shared file system. After the network traffic to the file system servers has stabilised, we start our application and its associated monitoring. Table~\ref{table: runtime_bkg} reports the corresponding execution time results. 

%\begin{table*}[!ht]
%\caption{}
%\centering
%%\resizebox{1\textwidth}{!}{\begin{minipage}{\textwidth}
%\begin{tabular}{  l  c  c  c  c  c  c }
%\toprule
% & \multicolumn{3}{c}{Lustre} & \multicolumn{3}{c}{GPFS} \\
%   \footnotesize & Exec time [sec] & \# reads & overhead [\%] & Exec time [sec] & \# reads & overhead [\%]\\
%   \midrule
%   w/o AM & 148.85$\pm$1.47 & 1615623$\pm$11111 & - & 104.35$\pm$1.48 & 4675$\pm$30 & - \\ 
%   w/ AM Tailore config & 81.82$\pm$1.17 & 1324290$\pm$4845 & 0.68 & 81.38$\pm$0.46 & 1089 & 3.06 \\ 
%   w/ AM WillNeed all file & 80.48$\pm$0.47 & 1320422$\pm$3476 & - & 79.72$\pm$0.41 & 762 & - \\ 
%\bottomrule  
%\end{tabular}
%\end{minipage}}
%\label{table: runtime_bkg}
%\end{table*} 

%%Since our advice infrastructure is meant to be used in everyday work, it's important to assess the overhead it introduces to the normal execution of the application.
%%We already saw that the improvement and the benefit can be big, both in the reduction of the execution time and of the number of reads the the storage system will have to handle.

%!TEX root = ../main.tex
\section{Related Work}
\label{sec: related}

Many research works have tried to optimise collective I/O focusing on different aspects. Yu and Vetter~\cite{WeikuanV08} before us have identified the global synchronisation problem as one of the most severe for collective I/O performance. They exploited access pattern characteristics, common in certain scientific workloads, to partition collective I/O into smaller communication groups and synchronise only within these. Block-tridiagonal patterns, not directly exploitable, are automatically reduced, through an intermediate file view, to a more manageable pattern and can thus take advantage of the proposed solution. The ADIOS library~\cite{CPE:CPE3125} addresses this problem similarly by dividing a single big file into multiple files to which collective I/O is carried out independently for separated smaller groups of processes. Lu, Chen, Thakur and Zhuang~\cite{YinYTY12} further explored collective I/O performance beyond global synchronisation and considered memory pressure of collective I/O buffers. They proposed a memory conscious implementation that accounts for reduced memory per core in future large scale systems. Liao~\cite{Liao11} focused on the file domain partitioning impact on parallel file systems' performance. He demonstrated that by choosing the right file domain partitioning strategy, matching the file system locking protocol, collective write performance can be greatly improved. Yong, Xian-He, Thakur, Roth and Gropp~\cite{YongXTRG11} addressed the problem of I/O server contention using a layout aware strategy to reorganize data in aggregators. On the same lines, Xuechen, Jiang and Davis~\cite{XuechenJD09} proposed a strategy to make collective I/O `resonant' by matching memory layout and physical placement of data in I/O servers and exploiting non-contiguous access primitives of PVFS2. The strategy proposed is similar in concept to the Lustre implementation of collective I/O in which file contiguous patterns are converted to stripe contiguous patterns and the concurrency level on OSTs can be set using the MPI-IO hint \codeword{romio\_lustre\_co\_ratio} (Client-OST ratio). Liu, Chen and Zhuang~\cite{JilianYY13} exploited the scheduling capabilities of PVFS2 I/O servers to rearrange I/O requests' order and better overlap read and shuffle phases among different processes. 

Lee, Ross, Thakur, Xiaosong and Winslett~\cite{LeeRTXW04} proposed RTS as infrastructure for remote file access and staging using MPI-IO. Similarly to our approach, RTS uses additional threads, Active Buffering Threads (ABT)~\cite{XiaosongWLS03}, to transfer data in background to the compute phase. Moreover, the authors also modified the ABT ROMIO driver implementation to stage data in the local file system whenever the amount of main memory runs low. Although they include collective I/O in their study, they lack a detailed evaluation of the impact that SSD caching can have on the different performance contributions of collective I/O and the additional reduction of memory pressure. Furthermore, remote staging of data requires additional nodes while we collocate storage with compute. The SCR library~\cite{SCR} also uses local storage resources to efficiently write checkpoint/restart data but this is targeted to a specific use case and requires the modification of the application's source code to be integrated. Other works, focus on I/O jitter reduction using multi-threading and local buffering resources~\cite{DorierACSO12}, but we do an evaluation of collective I/O and show how the effect of I/O jitter can become even more prominent when using fast NVM devices. More recently the Fast Forward I/O project~\cite{fastforward}, from U.S. Department of Energy (DOE), proposed a burst buffer architecture to absorb I/O bursts from file system clients into a small number of high performance storage proxies equipped with high-end solid state drives. This technique has been, e.g., implemented in the DDN Infinite Memory Engine~\cite{DDN}. Even though the burst buffer solution is interesting, it may require very expensive dedicated servers as well as significant changes to the storage system architecture. 

Unlike previous works, we proposed a fully integrated, prototype solution for new available memory technologies able to scale aggregate bandwidth in collective I/O with the number of available compute nodes. Additionally, our solution does not require any proprietary hardware or dedicated kit to work. We demonstrate that SSD based cache can reduce the synchronisation overhead intrinsic in the collective I/O implementation in ROMIO as well as the requirement for large collective buffers (memory pressure). Our implementation is compatible with legacy codes, since it does not require any change at the application level, and can work out of the box with any backend file system, although in DEEP-ER we focused on BeeGFS. At the moment the cache synchronisation is implemented in the ADIO UFS driver using pthreads. Future releases of BeeGFS will support native caching, including asynchronous flushing of local files to global file system. We have already integrated ROMIO with a BeeGFS driver that will take advantage of these functionalities.

%!TEX root = ../main.tex
\section{Conclusions and Future Work}
\label{sec: conclusion}

In this paper we have presented a new approach in using multi-tier storage systems to improve collective I/O performance in ROMIO. We have demonstrated that the usage of SSDs attached to compute nodes can improve collective I/O by increasing the aggregated I/O bandwidth, reducing the global synchronisation overhead, and finally the memory requirements. This can be done provided there is enough computation to hide background cache synchronisation costs.

At the moment local SSDs are not fully integrated in the storage hierarchy and researchers will have to figure out how to best exploit them. Burst buffers provide a possible solution but these require expensive dedicated kit, whereas we can use inexpensive commodity flash/solid state devices in computing nodes. In the short term we plan to test our SSD based approach with real applications from the DEEP-ER project, further validating the introduced benefits in real world workloads. In the longer term we plan to support cache reading operations and in general more complex and better integrated caching support and policies. In this direction, the Exascale10 initiative plans to build an Exascale ready middleware that will be able to efficiently support multiple integrated storage tiers in the I/O stack.


\section*{Acknowledgment}
This work has been funded by the FP7 program of the European Commission through the DEEP-ER project (Grant Agreement no. 610476) and supported by the Exascale1O (E10) initiative.

\bibliographystyle{IEEEtran}
\bibliography{bibliography}

\end{document}
